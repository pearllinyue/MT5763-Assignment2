\documentclass[]{article}
\usepackage{lmodern}
\usepackage{amssymb,amsmath}
\usepackage{ifxetex,ifluatex}
\usepackage{fixltx2e} % provides \textsubscript
\ifnum 0\ifxetex 1\fi\ifluatex 1\fi=0 % if pdftex
  \usepackage[T1]{fontenc}
  \usepackage[utf8]{inputenc}
\else % if luatex or xelatex
  \ifxetex
    \usepackage{mathspec}
  \else
    \usepackage{fontspec}
  \fi
  \defaultfontfeatures{Ligatures=TeX,Scale=MatchLowercase}
\fi
% use upquote if available, for straight quotes in verbatim environments
\IfFileExists{upquote.sty}{\usepackage{upquote}}{}
% use microtype if available
\IfFileExists{microtype.sty}{%
\usepackage{microtype}
\UseMicrotypeSet[protrusion]{basicmath} % disable protrusion for tt fonts
}{}
\usepackage[margin=1in]{geometry}
\usepackage{hyperref}
\hypersetup{unicode=true,
            pdftitle={MT5763-Project2 Report},
            pdfauthor={Student ID: 180025784},
            pdfborder={0 0 0},
            breaklinks=true}
\urlstyle{same}  % don't use monospace font for urls
\usepackage{color}
\usepackage{fancyvrb}
\newcommand{\VerbBar}{|}
\newcommand{\VERB}{\Verb[commandchars=\\\{\}]}
\DefineVerbatimEnvironment{Highlighting}{Verbatim}{commandchars=\\\{\}}
% Add ',fontsize=\small' for more characters per line
\usepackage{framed}
\definecolor{shadecolor}{RGB}{248,248,248}
\newenvironment{Shaded}{\begin{snugshade}}{\end{snugshade}}
\newcommand{\KeywordTok}[1]{\textcolor[rgb]{0.13,0.29,0.53}{\textbf{#1}}}
\newcommand{\DataTypeTok}[1]{\textcolor[rgb]{0.13,0.29,0.53}{#1}}
\newcommand{\DecValTok}[1]{\textcolor[rgb]{0.00,0.00,0.81}{#1}}
\newcommand{\BaseNTok}[1]{\textcolor[rgb]{0.00,0.00,0.81}{#1}}
\newcommand{\FloatTok}[1]{\textcolor[rgb]{0.00,0.00,0.81}{#1}}
\newcommand{\ConstantTok}[1]{\textcolor[rgb]{0.00,0.00,0.00}{#1}}
\newcommand{\CharTok}[1]{\textcolor[rgb]{0.31,0.60,0.02}{#1}}
\newcommand{\SpecialCharTok}[1]{\textcolor[rgb]{0.00,0.00,0.00}{#1}}
\newcommand{\StringTok}[1]{\textcolor[rgb]{0.31,0.60,0.02}{#1}}
\newcommand{\VerbatimStringTok}[1]{\textcolor[rgb]{0.31,0.60,0.02}{#1}}
\newcommand{\SpecialStringTok}[1]{\textcolor[rgb]{0.31,0.60,0.02}{#1}}
\newcommand{\ImportTok}[1]{#1}
\newcommand{\CommentTok}[1]{\textcolor[rgb]{0.56,0.35,0.01}{\textit{#1}}}
\newcommand{\DocumentationTok}[1]{\textcolor[rgb]{0.56,0.35,0.01}{\textbf{\textit{#1}}}}
\newcommand{\AnnotationTok}[1]{\textcolor[rgb]{0.56,0.35,0.01}{\textbf{\textit{#1}}}}
\newcommand{\CommentVarTok}[1]{\textcolor[rgb]{0.56,0.35,0.01}{\textbf{\textit{#1}}}}
\newcommand{\OtherTok}[1]{\textcolor[rgb]{0.56,0.35,0.01}{#1}}
\newcommand{\FunctionTok}[1]{\textcolor[rgb]{0.00,0.00,0.00}{#1}}
\newcommand{\VariableTok}[1]{\textcolor[rgb]{0.00,0.00,0.00}{#1}}
\newcommand{\ControlFlowTok}[1]{\textcolor[rgb]{0.13,0.29,0.53}{\textbf{#1}}}
\newcommand{\OperatorTok}[1]{\textcolor[rgb]{0.81,0.36,0.00}{\textbf{#1}}}
\newcommand{\BuiltInTok}[1]{#1}
\newcommand{\ExtensionTok}[1]{#1}
\newcommand{\PreprocessorTok}[1]{\textcolor[rgb]{0.56,0.35,0.01}{\textit{#1}}}
\newcommand{\AttributeTok}[1]{\textcolor[rgb]{0.77,0.63,0.00}{#1}}
\newcommand{\RegionMarkerTok}[1]{#1}
\newcommand{\InformationTok}[1]{\textcolor[rgb]{0.56,0.35,0.01}{\textbf{\textit{#1}}}}
\newcommand{\WarningTok}[1]{\textcolor[rgb]{0.56,0.35,0.01}{\textbf{\textit{#1}}}}
\newcommand{\AlertTok}[1]{\textcolor[rgb]{0.94,0.16,0.16}{#1}}
\newcommand{\ErrorTok}[1]{\textcolor[rgb]{0.64,0.00,0.00}{\textbf{#1}}}
\newcommand{\NormalTok}[1]{#1}
\usepackage{longtable,booktabs}
\usepackage{graphicx,grffile}
\makeatletter
\def\maxwidth{\ifdim\Gin@nat@width>\linewidth\linewidth\else\Gin@nat@width\fi}
\def\maxheight{\ifdim\Gin@nat@height>\textheight\textheight\else\Gin@nat@height\fi}
\makeatother
% Scale images if necessary, so that they will not overflow the page
% margins by default, and it is still possible to overwrite the defaults
% using explicit options in \includegraphics[width, height, ...]{}
\setkeys{Gin}{width=\maxwidth,height=\maxheight,keepaspectratio}
\IfFileExists{parskip.sty}{%
\usepackage{parskip}
}{% else
\setlength{\parindent}{0pt}
\setlength{\parskip}{6pt plus 2pt minus 1pt}
}
\setlength{\emergencystretch}{3em}  % prevent overfull lines
\providecommand{\tightlist}{%
  \setlength{\itemsep}{0pt}\setlength{\parskip}{0pt}}
\setcounter{secnumdepth}{0}
% Redefines (sub)paragraphs to behave more like sections
\ifx\paragraph\undefined\else
\let\oldparagraph\paragraph
\renewcommand{\paragraph}[1]{\oldparagraph{#1}\mbox{}}
\fi
\ifx\subparagraph\undefined\else
\let\oldsubparagraph\subparagraph
\renewcommand{\subparagraph}[1]{\oldsubparagraph{#1}\mbox{}}
\fi

%%% Use protect on footnotes to avoid problems with footnotes in titles
\let\rmarkdownfootnote\footnote%
\def\footnote{\protect\rmarkdownfootnote}

%%% Change title format to be more compact
\usepackage{titling}

% Create subtitle command for use in maketitle
\newcommand{\subtitle}[1]{
  \posttitle{
    \begin{center}\large#1\end{center}
    }
}

\setlength{\droptitle}{-2em}

  \title{MT5763-Project2 Report}
    \pretitle{\vspace{\droptitle}\centering\huge}
  \posttitle{\par}
    \author{Student ID: 180025784}
    \preauthor{\centering\large\emph}
  \postauthor{\par}
      \predate{\centering\large\emph}
  \postdate{\par}
    \date{Due 5 Novmber 2018}


\begin{document}
\maketitle

{
\setcounter{tocdepth}{3}
\tableofcontents
}
\subsection{Executive Summary}\label{executive-summary}

The report to build a linear model that predicts Oxygen intake rates,
which is a measure of aerobic fitness. The data used in this report is
from the study of Rawlings (1998), \emph{Applied Regression Analysis: A
Research Tool} 2nd Edition. The final model was fitted using linear
regression, diagnosed using ANOVA, AIC, qqnorm, qqline, shapiro test,
ncvtest, Durbin Watson Test and updatad to a good one. The confidence
intervals are provided by the bootstrapping function written by the
group work of all members of Drunken Master2. The randomisation test is
also carried out. The final model states that the Oxygen intake rates
has a strong relationship with age, weight, runtime, restpulse and
maxpulse. \pagebreak

\subsection{Introduction}\label{introduction}

The improvement of bootstrap function completed by Drunken Master2
provided the basic code for the conduct of this project. The data used
by this project is provided by Rawlings (1998).

The purpose of the investigation of this report is to seek the best
model that predicts Oxygen intake rates based on a list of measurements
which can be obtained easily. All the data includes the following seven
variables from 31 male samples as below:

• Age: Age in years of 31 samples

• Weight: Weight in kg of 31 samples

• Oxygen: Oxygen intake rate in \emph{ml}/kg body weight per minute

• RunTime: time for each sample to run 1.5 miles in minutes

• RestPulse: heart rate when having a rest

• RunPulse: heart rate after running

• MaxPulse: maximum heart rate during the run

The intended audience of this report should be those are statistically
literate, but not familiar with computer-intensive inference. The
conduct of this project are all used R 3.5.1 (R, 2018).

\pagebreak

\subsection{Methods}\label{methods}

Due to the fact that the target audience of this report are not familiar
with computer-intensive inference, the methods mainly contains the
interpretation of bootstrapping and randomisation. The aim of these is
to carry out effectively statistical simulations.

\subsubsection{1 Bootstrapping function for Confidence
Intervals}\label{bootstrapping-function-for-confidence-intervals}

Bootstrapping method is a uniform sample that is put back from a given
training set, in other words, whenever a sample is selected, it may be
selected again and added to the training set again. The method for
providing confidence intervals was present by Bradley Efron (1987). For
small data set,bootstrapping works well. Boostrapping generates
approximate sampling distributions of parameters to get confidence
intervals

\subsubsection{2 Randomisation Tests}\label{randomisation-tests}

In randomisation tests, \emph{H0} refers to null hypothesis whilst
\emph{H1} represents alternative hypothesis. Randomisation tests
generates parameter distributions assuming if \emph{H0} is true and can
get \emph{p}-values. Actually, there are three main ways of
randomisation \emph{t}-test as below:

• Comparison of means of two sampled populations

\emph{H0}: the means of two populations are equal

\emph{H1}: the means of two populations are not equal

• Comparison of means of a sampled population that of some hypothesised
value

\emph{H0}: the means of the population is equal to theoretial constant

\emph{H1}: the means of two populations is equal to theoretial constant

• Comparison of the variant for analysing paired observations

\emph{H0}: the slope of the relationship between explanatory variable
and response variable is zero - no linear association

\emph{H1}: the slope of the relationship between explanatory variable
and response variable is not zero - linear association exists

\pagebreak

\subsection{Results}\label{results}

\subsubsection{1 Model Fitting}\label{model-fitting}

The initial model was fitted with all other variables, including Age,
Weight, RunTime, RestPulse and MaxPulse. The coefficients of these
variables regarding to Oxygen can be seen in Figure 1.

\begin{longtable}[]{@{}ll@{}}
\caption{Figure 1: Coefficients of the variables in the
model}\tabularnewline
\toprule
Variable & Coefficient\tabularnewline
\midrule
\endfirsthead
\toprule
Variable & Coefficient\tabularnewline
\midrule
\endhead
(Intercept) & 102.934\tabularnewline
Age & -0.227\tabularnewline
Weight & -0.074\tabularnewline
RunTime & -2.629\tabularnewline
RestPulse & -0.022\tabularnewline
RunPulse & -0.37\tabularnewline
MaxPulse & 0.303\tabularnewline
\bottomrule
\end{longtable}

Further analysis method ANOVA was used for initial model as in Figure 2.
In Figure 2, we can see the p-value of each variable regarding to
Oxygen. It is clear that RunTime has the most strongly significant
correlation with Oxygen while RestPulse and Weight coule be not be very
related to Oxygen. Thus, these two variables would probably be abandoned
when we better the model. Also, RunPulse has a comparatively strongly
significant correlation and MaxPulse and Age also affect Oxygen.

\begin{longtable}[]{@{}ll@{}}
\caption{Figure 2: ANOVA of the initial model}\tabularnewline
\toprule
Variable & p-value\tabularnewline
\midrule
\endfirsthead
\toprule
Variable & p-value\tabularnewline
\midrule
\endhead
Age & 0.032\tabularnewline
Weight & 0.187\tabularnewline
RunTime & 0\tabularnewline
RestPulse & 0.747\tabularnewline
RunPulse & 0.005\tabularnewline
MaxPulse & 0.036\tabularnewline
\bottomrule
\end{longtable}

Generally, Akaike Information Criterion (AIC) was used to select several
fitted model objects to update the initial model. AIC method offered two
models, one contains all six variables whose AIC score is 58.16, another
one contains five variables (Age, Weight, RunTime, RunPulse and
MaxPulse) whose AIC score is 56.3. Thus, AIC method automatically chose
the latter one to be the new model. Then ANOVA method would be used
again to check the variables of new model and see the p-value of each
variable as in Figure 3. Although Weight still not seems to be
sigificantly relevant with Oxygen because its p-value is more than 0.05,
the other variables all shows inordinately to be related to Oxygen,
which can be inferred that the new model is much better than the initial
one.

\begin{longtable}[]{@{}ll@{}}
\caption{Figure 3: ANOVA of the new model}\tabularnewline
\toprule
Variable & p-value\tabularnewline
\midrule
\endfirsthead
\toprule
Variable & p-value\tabularnewline
\midrule
\endhead
Age & 0.03\tabularnewline
Weight & 0.187\tabularnewline
RunTime & 0\tabularnewline
RunPulse & 0.004\tabularnewline
MaxPulse & 0.032\tabularnewline
\bottomrule
\end{longtable}

The model diagnostics was started to check the shape of errors by
distribution of residuals.The QQ plot as well as the Shapiro-Wilks test
was used in this report. The QQ plot and QQ norm can be seen in Figure 4
which displays the data set fits this model pretty good. Also, the
Shapiro-Wilks test shows that w = 0.92131, which means the data fits
well with the normal distribution in that the closer w is to 1, the
model fits better with the normal distribution.

Figure 4: QQ plot of new model

\includegraphics{180025784-MT5763-Project2_files/figure-latex/qqplot for model after AIC-1.pdf}

In addtion, the histogram of residuals of the new model can be seen in
Figure 5. We can also track down the extreme residuals which shows that
data in Row 15 and 17 are having extreme residuals.

Figure 5: histogram of residuals of new model

\includegraphics{180025784-MT5763-Project2_files/figure-latex/hist for model after AIC-1.pdf}

The next step of model diagnostics is error distribution, including
check variance of residuals and Breusch-Pagan test. The error spread of
the new model can be plotted in Figure 6. The output of ncvTest
(i.e.~Breusch-Pagan test) is p = 0.77806, which means to choose
\emph{H1} that the error variance varies with the level of the fitted
value.

Figure 6: plot of residuals spread

\includegraphics{180025784-MT5763-Project2_files/figure-latex/plot of residuals spread-1.pdf}

The following step is to check serial correlation of residuals by using
Durbin-Watson test. The \emph{p}-value of this test is 0.374
\textgreater{} 0.05, which means the errors are correlated.

And the final step is to check collinearity of the model. Figure 7 shows
the dependencies between covariates and we use this to update the model
into a new altered model. In additionl, Variance Inflation Factor (VIF)
was used to check if multicollinearity exists between variables in
Figure 8.

Figure 7: Dependencies between covarites

\begin{verbatim}
## Loading required package: ggplot2
\end{verbatim}

\begin{verbatim}
## -- Attaching packages -------------------------------------------- tidyverse 1.2.1 --
\end{verbatim}

\begin{verbatim}
## √ tibble  1.4.2     √ purrr   0.2.5
## √ tidyr   0.8.1     √ dplyr   0.7.6
## √ readr   1.1.1     √ stringr 1.3.1
## √ tibble  1.4.2     √ forcats 0.3.0
\end{verbatim}

\begin{verbatim}
## -- Conflicts ----------------------------------------------- tidyverse_conflicts() --
## x dplyr::filter() masks stats::filter()
## x dplyr::lag()    masks stats::lag()
\end{verbatim}

\includegraphics{180025784-MT5763-Project2_files/figure-latex/dependencies between covarites-1.pdf}

\begin{longtable}[]{@{}ll@{}}
\caption{Figure 8: VIF of altered Model}\tabularnewline
\toprule
Variable & VIF\tabularnewline
\midrule
\endfirsthead
\toprule
Variable & VIF\tabularnewline
\midrule
\endhead
Age & 1.436\tabularnewline
Weight & 1.143\tabularnewline
RunTime & 1.297\tabularnewline
RunPulse & 8.359\tabularnewline
MaxPulse & 8.731\tabularnewline
\bottomrule
\end{longtable}

Then the final model was altered and updated through these procedures.
Then QQ norm, Shapiro-Wilks Test and ncvTest would be carried out again
to see if there are any changes. The QQ norm is plotted in Figure 9
below. Plus, the Shapiro-Wilks Test is 0.96627, which is closer to 1
than the previous 0.92131, which means the data fits much better with
the normal distribution. And the p-value of ncvTest is 0.43469.

Figure 9: QQ norm of altered Model

\includegraphics{180025784-MT5763-Project2_files/figure-latex/QQ norm of alteredModel-1.pdf}

The coefficients and confidence intervals of the final model can be seen
in Figure 10.

\begin{longtable}[]{@{}llll@{}}
\caption{Figure 10: Coefficients and Confidence Intervals of final
Model}\tabularnewline
\toprule
Variable & Coefficient & 2.5\% CI & 97.5\% CI\tabularnewline
\midrule
\endfirsthead
\toprule
Variable & Coefficient & 2.5\% CI & 97.5\% CI\tabularnewline
\midrule
\endhead
(Intercept) & 102.204 & 77.532 & 126.876\tabularnewline
Age & -0.22 & -0.416 & -0.223\tabularnewline
Weight & -0.072 & -0.182 & 0.037\tabularnewline
RunTime & -2.683 & -3.385 & -1.98\tabularnewline
RunPulse & -0.373 & -0.615 & -0.132\tabularnewline
MaxPulse & 0.305 & 0.029 & 0.581\tabularnewline
\bottomrule
\end{longtable}

\pagebreak

\subsubsection{2 Bootstrapping for Confidence
Intervals}\label{bootstrapping-for-confidence-intervals}

The bootstrapping function was from the group work of Drunken Master2.
And the confidence intervals of each variable of the final model can be
obtained as below in Figure 11.

\begin{longtable}[]{@{}lll@{}}
\caption{Figure 11: Confidence Intervals provided by bootstrapping
function of final Model}\tabularnewline
\toprule
Variable & 2.5\% CI & 97.5\% CI\tabularnewline
\midrule
\endfirsthead
\toprule
Variable & 2.5\% CI & 97.5\% CI\tabularnewline
\midrule
\endhead
(Intercept) & 89 & 118.2\tabularnewline
Age & -0.3698 & -0.0713\tabularnewline
Weight & -0.1631 & -0.026\tabularnewline
RunTime & -2.915 & -2.071\tabularnewline
RunPulse & -0.5703 & -0.1909\tabularnewline
MaxPulse & 0.1872 & 0.527\tabularnewline
\bottomrule
\end{longtable}

\pagebreak

\subsubsection{3 Randomisastion Tests}\label{randomisastion-tests}

In order to perform randmisation test for all model terms, I chose the
test for a slope parameter to see if there is a linear association
between explanatory variables and the response variable in the final
model. Therefore, \emph{H0} represents the slope of the relationship is
zero, that is, there is no linear association whilst \emph{H1}
represents the slope of the relationship is not zero, that is, there is
linear association. The result of this can be shown in the form of
histogram with abline in Figure 12.

Figure 12: Histogram and line of estimatedSlope

\includegraphics{180025784-MT5763-Project2_files/figure-latex/Histogram and line of estimatedSlope-1.pdf}

From the figure, it can be observed that data is a little bit extreme if
\emph{H0} is true. Thus, maybe we should reject \emph{H0}. To be more
precise, I calculated the the minimum of \emph{k}/1000 and
1-\emph{k}/1000, where \emph{k} is the ordered position of original
sample estimate. The output of this is 0.31, which means p
\textless{}0.31. Therefore, it is hard to say that \emph{H0} can be
accepted. We can say from this that there is linear association in the
final model.

\pagebreak

\subsection{Discussion}\label{discussion}

The model was fitted and updated to predict Oxygen intake rates on the
basis of a series of other measurements. The final model is
alteredModel. This model states that the parameters affecting Oxygen
intake rates are Age, Weight, RunTime, RunPulse and MaxPulse.

The effects of the variables in alteredModel on Oxygen intake rates are
as below:

• As the age increases by 1 year, oxygen intake rates decreases by
0.2196 ml/kg per minute

• As the weight increases by 1 kg, oxygen intake rates decreases by
0.0723 ml/kg per minute

• As the runtime increases by 1 unit, oxygen intake rates decreases by
2.6825 ml/kg per minute

• As the runpulse increases by 1 unit, oxygen intake rates decreases by
0.3734 ml/kg per minute

• As the maxpulse increases by 1 unit, oxygen intake rates increases by
0.3049 ml/kg per minute

It can be concluded from the model that elder people and heavier people
will intake less oxygen rate. Also, the less the runtime and runpulse
are, the less oxygen intake rates is. However, as the maxpulse
increases, oxygen intake rates will increase, which is the only variable
that makes oxygen intake rates high.

Additionally, the partial relationships between each covariate and the
response can be found in Appendix.

\pagebreak

\subsection{References}\label{references}

Donovan, C. (2018) MT5763 Project 2 - code collaboration and computer
intensive inference. {[}Online{]}

Efron, B. (1987) `Better bootstrap confidence intervals', \emph{Journal
of the American Statistical Association}, 82(397), pp.~171-385.
Available
at:\url{https://www.jstor.org/stable/pdf/2289144.pdf?casa_token=4iZ6QKczb9UAAAAA:I4RQeCRasPcZkONB3D1TYhkVgOVnzqEsXLW9-hjn-n8rhEPzaW5cRm2kIi28XI00vyGc3kfo_87ThMqvfvSRjdkGsYoOyMQhLFaNXm3Rsv_pKBNGRMc/}
(Accessed: 3 Nov 2018)

Rawlings, J.O. (1998) \emph{Applied regression analysis: a research
tool} (2nd Edition). Berlin: Springer.

R Core Team (2018) \emph{R: A language and environment for statistical
computing}. R Foundation for Statistical Computing, Vienna, Austria.
Available at: \url{https://www.R-project.org/} (Accessed: 2 Nov 2018)

\pagebreak

\subsection{Appendices}\label{appendices}

\subsubsection{\texorpdfstring{Appendix 1 \emph{R code and output for
model
fitting}}{Appendix 1 R code and output for model fitting}}\label{appendix-1-r-code-and-output-for-model-fitting}

\begin{Shaded}
\begin{Highlighting}[]
\CommentTok{#input data}
\NormalTok{fitness <-}\StringTok{ }\KeywordTok{read.csv}\NormalTok{(}\StringTok{"/Users/apple/Desktop/MT5763/Assignment 2/fitness.csv"}\NormalTok{, }\DataTypeTok{header =}\NormalTok{ T)}
\KeywordTok{head}\NormalTok{(fitness)}
\end{Highlighting}
\end{Shaded}

\begin{verbatim}
##   Age Weight Oxygen RunTime RestPulse RunPulse MaxPulse
## 1  44  89.47 44.609   11.37        62      178      182
## 2  40  75.07 45.313   10.07        62      185      185
## 3  44  85.84 54.297    8.65        45      156      168
## 4  42  68.15 59.571    8.17        40      166      172
## 5  38  89.02 49.874    9.22        55      178      180
## 6  47  77.45 44.811   11.63        58      176      176
\end{verbatim}

\begin{Shaded}
\begin{Highlighting}[]
\CommentTok{#fit an initial model "Oxygen_Model" to predict "Oxygen"}
\NormalTok{Oxygen_Model <-}\StringTok{ }\KeywordTok{lm}\NormalTok{(Oxygen}\OperatorTok{~}\NormalTok{., }\DataTypeTok{data =}\NormalTok{ fitness)}
\KeywordTok{summary}\NormalTok{(Oxygen_Model)}
\end{Highlighting}
\end{Shaded}

\begin{verbatim}
## 
## Call:
## lm(formula = Oxygen ~ ., data = fitness)
## 
## Residuals:
##     Min      1Q  Median      3Q     Max 
## -5.4026 -0.8991  0.0706  1.0496  5.3847 
## 
## Coefficients:
##              Estimate Std. Error t value Pr(>|t|)    
## (Intercept) 102.93448   12.40326   8.299 1.64e-08 ***
## Age          -0.22697    0.09984  -2.273  0.03224 *  
## Weight       -0.07418    0.05459  -1.359  0.18687    
## RunTime      -2.62865    0.38456  -6.835 4.54e-07 ***
## RestPulse    -0.02153    0.06605  -0.326  0.74725    
## RunPulse     -0.36963    0.11985  -3.084  0.00508 ** 
## MaxPulse      0.30322    0.13650   2.221  0.03601 *  
## ---
## Signif. codes:  0 '***' 0.001 '**' 0.01 '*' 0.05 '.' 0.1 ' ' 1
## 
## Residual standard error: 2.317 on 24 degrees of freedom
## Multiple R-squared:  0.8487, Adjusted R-squared:  0.8108 
## F-statistic: 22.43 on 6 and 24 DF,  p-value: 9.715e-09
\end{verbatim}

\begin{Shaded}
\begin{Highlighting}[]
\CommentTok{#anova of the initial model}
\CommentTok{#install.packages("car")}
\KeywordTok{library}\NormalTok{(car)}
\end{Highlighting}
\end{Shaded}

\begin{verbatim}
## Loading required package: carData
\end{verbatim}

\begin{verbatim}
## 
## Attaching package: 'car'
\end{verbatim}

\begin{verbatim}
## The following object is masked from 'package:dplyr':
## 
##     recode
\end{verbatim}

\begin{verbatim}
## The following object is masked from 'package:purrr':
## 
##     some
\end{verbatim}

\begin{Shaded}
\begin{Highlighting}[]
\KeywordTok{Anova}\NormalTok{(Oxygen_Model)}
\end{Highlighting}
\end{Shaded}

\begin{verbatim}
## Anova Table (Type II tests)
## 
## Response: Oxygen
##            Sum Sq Df F value    Pr(>F)    
## Age        27.746  1  5.1685  0.032235 *  
## Weight      9.911  1  1.8461  0.186866    
## RunTime   250.822  1 46.7233 4.538e-07 ***
## RestPulse   0.571  1  0.1063  0.747250    
## RunPulse   51.058  1  9.5111  0.005079 ** 
## MaxPulse   26.491  1  4.9348  0.036007 *  
## Residuals 128.838 24                      
## ---
## Signif. codes:  0 '***' 0.001 '**' 0.01 '*' 0.05 '.' 0.1 ' ' 1
\end{verbatim}

\begin{Shaded}
\begin{Highlighting}[]
\CommentTok{#choose a model by AIC and do anova for the new model}
\NormalTok{Oxygen_Model <-}\StringTok{ }\KeywordTok{step}\NormalTok{(Oxygen_Model)}
\end{Highlighting}
\end{Shaded}

\begin{verbatim}
## Start:  AIC=58.16
## Oxygen ~ Age + Weight + RunTime + RestPulse + RunPulse + MaxPulse
## 
##             Df Sum of Sq    RSS    AIC
## - RestPulse  1     0.571 129.41 56.299
## <none>                   128.84 58.162
## - Weight     1     9.911 138.75 58.459
## - MaxPulse   1    26.491 155.33 61.958
## - Age        1    27.746 156.58 62.208
## - RunPulse   1    51.058 179.90 66.510
## - RunTime    1   250.822 379.66 89.664
## 
## Step:  AIC=56.3
## Oxygen ~ Age + Weight + RunTime + RunPulse + MaxPulse
## 
##            Df Sum of Sq    RSS    AIC
## <none>                  129.41 56.299
## - Weight    1      9.52 138.93 56.499
## - MaxPulse  1     26.83 156.23 60.139
## - Age       1     27.37 156.78 60.247
## - RunPulse  1     52.60 182.00 64.871
## - RunTime   1    320.36 449.77 92.917
\end{verbatim}

\begin{Shaded}
\begin{Highlighting}[]
\KeywordTok{Anova}\NormalTok{(Oxygen_Model)}
\end{Highlighting}
\end{Shaded}

\begin{verbatim}
## Anova Table (Type II tests)
## 
## Response: Oxygen
##           Sum Sq Df F value    Pr(>F)    
## Age        27.37  1  5.2884   0.03010 *  
## Weight      9.52  1  1.8394   0.18714    
## RunTime   320.36  1 61.8892 3.186e-08 ***
## RunPulse   52.60  1 10.1609   0.00383 ** 
## MaxPulse   26.83  1  5.1825   0.03164 *  
## Residuals 129.41 25                      
## ---
## Signif. codes:  0 '***' 0.001 '**' 0.01 '*' 0.05 '.' 0.1 ' ' 1
\end{verbatim}

\begin{Shaded}
\begin{Highlighting}[]
\CommentTok{#model diagnostics & remedial actions}

\NormalTok{##model diagnostics - error shape}
\KeywordTok{qqnorm}\NormalTok{(}\KeywordTok{resid}\NormalTok{(Oxygen_Model))}
\KeywordTok{qqline}\NormalTok{(}\KeywordTok{resid}\NormalTok{(Oxygen_Model))}
\end{Highlighting}
\end{Shaded}

\includegraphics{180025784-MT5763-Project2_files/figure-latex/code for model fitting-1.pdf}

\begin{Shaded}
\begin{Highlighting}[]
\KeywordTok{shapiro.test}\NormalTok{(}\KeywordTok{resid}\NormalTok{(Oxygen_Model))}
\end{Highlighting}
\end{Shaded}

\begin{verbatim}
## 
##  Shapiro-Wilk normality test
## 
## data:  resid(Oxygen_Model)
## W = 0.96627, p-value = 0.4227
\end{verbatim}

\begin{Shaded}
\begin{Highlighting}[]
\KeywordTok{hist}\NormalTok{(}\KeywordTok{resid}\NormalTok{(Oxygen_Model))}
\end{Highlighting}
\end{Shaded}

\includegraphics{180025784-MT5763-Project2_files/figure-latex/code for model fitting-2.pdf}

\begin{Shaded}
\begin{Highlighting}[]
\NormalTok{###track down the extreme residuals}
\NormalTok{bigResid <-}\StringTok{ }\KeywordTok{which}\NormalTok{(}\KeywordTok{abs}\NormalTok{(}\KeywordTok{resid}\NormalTok{(Oxygen_Model))}\OperatorTok{>}\DecValTok{5}\NormalTok{)}
\NormalTok{fitness[bigResid,]}
\end{Highlighting}
\end{Shaded}

\begin{verbatim}
##    Age Weight Oxygen RunTime RestPulse RunPulse MaxPulse
## 15  54  83.12 51.855   10.33        50      166      170
## 17  51  69.63 40.836   10.95        57      168      172
\end{verbatim}

\begin{Shaded}
\begin{Highlighting}[]
\NormalTok{##model diagnostics - error spread}
\NormalTok{Oxygen_Resid <-}\StringTok{ }\KeywordTok{resid}\NormalTok{(Oxygen_Model)}
\KeywordTok{plot}\NormalTok{(}\KeywordTok{fitted}\NormalTok{(Oxygen_Model), Oxygen_Resid, }\DataTypeTok{ylab =} \StringTok{'residuals'}\NormalTok{, }\DataTypeTok{xlab =} \StringTok{'Fitted values'}\NormalTok{)}
\end{Highlighting}
\end{Shaded}

\includegraphics{180025784-MT5763-Project2_files/figure-latex/code for model fitting-3.pdf}

\begin{Shaded}
\begin{Highlighting}[]
\NormalTok{###testing using the ncvTest}
\KeywordTok{ncvTest}\NormalTok{(Oxygen_Model, }\OperatorTok{~}\NormalTok{.)}
\end{Highlighting}
\end{Shaded}

\begin{verbatim}
## Non-constant Variance Score Test 
## Variance formula: ~ . 
## Chisquare = 4.014893, Df = 7, p = 0.77806
\end{verbatim}

\begin{Shaded}
\begin{Highlighting}[]
\NormalTok{##model diagnostics - error independence (durbinWatsonTest)}
\KeywordTok{durbinWatsonTest}\NormalTok{(Oxygen_Model)}
\end{Highlighting}
\end{Shaded}

\begin{verbatim}
##  lag Autocorrelation D-W Statistic p-value
##    1       0.1413284      1.690357   0.374
##  Alternative hypothesis: rho != 0
\end{verbatim}

\begin{Shaded}
\begin{Highlighting}[]
\NormalTok{##model diagnostics - default plots}
\KeywordTok{plot}\NormalTok{(Oxygen_Model, }\DataTypeTok{which =} \DecValTok{1}\OperatorTok{:}\DecValTok{2}\NormalTok{)}
\end{Highlighting}
\end{Shaded}

\includegraphics{180025784-MT5763-Project2_files/figure-latex/code for model fitting-4.pdf}
\includegraphics{180025784-MT5763-Project2_files/figure-latex/code for model fitting-5.pdf}

\begin{Shaded}
\begin{Highlighting}[]
\NormalTok{##model diagnostics - collinearity}
\KeywordTok{library}\NormalTok{(GGally)}
\KeywordTok{library}\NormalTok{(tidyverse)}
\NormalTok{numericOnly <-}\StringTok{ }\NormalTok{fitness }\OperatorTok\StringTok{ }\KeywordTok{select_if}\NormalTok{(is.numeric)}
\KeywordTok{ggpairs}\NormalTok{(numericOnly)}
\end{Highlighting}
\end{Shaded}

\includegraphics{180025784-MT5763-Project2_files/figure-latex/code for model fitting-6.pdf}

\begin{Shaded}
\begin{Highlighting}[]
\NormalTok{##make a better altered model}
\NormalTok{alteredModel <-}\StringTok{ }\KeywordTok{update}\NormalTok{(Oxygen_Model,.}\OperatorTok{~}\NormalTok{.}\OperatorTok{-}\NormalTok{comp.ratio)}
\KeywordTok{vif}\NormalTok{(alteredModel)}
\end{Highlighting}
\end{Shaded}

\begin{verbatim}
##      Age   Weight  RunTime RunPulse MaxPulse 
## 1.435634 1.142505 1.297127 8.358594 8.731225
\end{verbatim}

\begin{Shaded}
\begin{Highlighting}[]
\NormalTok{##again: model diagnostics - error shape}
\KeywordTok{qqnorm}\NormalTok{(}\KeywordTok{resid}\NormalTok{(alteredModel))}
\end{Highlighting}
\end{Shaded}

\includegraphics{180025784-MT5763-Project2_files/figure-latex/code for model fitting-7.pdf}

\begin{Shaded}
\begin{Highlighting}[]
\KeywordTok{shapiro.test}\NormalTok{((}\KeywordTok{resid}\NormalTok{(alteredModel)))}
\end{Highlighting}
\end{Shaded}

\begin{verbatim}
## 
##  Shapiro-Wilk normality test
## 
## data:  (resid(alteredModel))
## W = 0.96627, p-value = 0.4227
\end{verbatim}

\begin{Shaded}
\begin{Highlighting}[]
\KeywordTok{ncvTest}\NormalTok{(alteredModel)}
\end{Highlighting}
\end{Shaded}

\begin{verbatim}
## Non-constant Variance Score Test 
## Variance formula: ~ fitted.values 
## Chisquare = 0.6102665, Df = 1, p = 0.43469
\end{verbatim}

\begin{Shaded}
\begin{Highlighting}[]
\NormalTok{##summary of altered model}
\KeywordTok{summary}\NormalTok{(alteredModel)}
\end{Highlighting}
\end{Shaded}

\begin{verbatim}
## 
## Call:
## lm(formula = Oxygen ~ Age + Weight + RunTime + RunPulse + MaxPulse, 
##     data = fitness)
## 
## Residuals:
##     Min      1Q  Median      3Q     Max 
## -5.4724 -0.8476  0.0094  0.9976  5.3807 
## 
## Coefficients:
##              Estimate Std. Error t value Pr(>|t|)    
## (Intercept) 102.20428   11.97929   8.532 7.13e-09 ***
## Age          -0.21962    0.09550  -2.300  0.03010 *  
## Weight       -0.07230    0.05331  -1.356  0.18714    
## RunTime      -2.68252    0.34099  -7.867 3.19e-08 ***
## RunPulse     -0.37340    0.11714  -3.188  0.00383 ** 
## MaxPulse      0.30491    0.13394   2.277  0.03164 *  
## ---
## Signif. codes:  0 '***' 0.001 '**' 0.01 '*' 0.05 '.' 0.1 ' ' 1
## 
## Residual standard error: 2.275 on 25 degrees of freedom
## Multiple R-squared:  0.848,  Adjusted R-squared:  0.8176 
## F-statistic:  27.9 on 5 and 25 DF,  p-value: 1.811e-09
\end{verbatim}

\begin{Shaded}
\begin{Highlighting}[]
\NormalTok{##anova and confidence intervals of altered model}
\KeywordTok{Anova}\NormalTok{(alteredModel)}
\end{Highlighting}
\end{Shaded}

\begin{verbatim}
## Anova Table (Type II tests)
## 
## Response: Oxygen
##           Sum Sq Df F value    Pr(>F)    
## Age        27.37  1  5.2884   0.03010 *  
## Weight      9.52  1  1.8394   0.18714    
## RunTime   320.36  1 61.8892 3.186e-08 ***
## RunPulse   52.60  1 10.1609   0.00383 ** 
## MaxPulse   26.83  1  5.1825   0.03164 *  
## Residuals 129.41 25                      
## ---
## Signif. codes:  0 '***' 0.001 '**' 0.01 '*' 0.05 '.' 0.1 ' ' 1
\end{verbatim}

\begin{Shaded}
\begin{Highlighting}[]
\KeywordTok{confint}\NormalTok{(alteredModel)}
\end{Highlighting}
\end{Shaded}

\begin{verbatim}
##                   2.5 %       97.5 %
## (Intercept) 77.53246619 126.87608420
## Age         -0.41631235  -0.02293041
## Weight      -0.18209653   0.03749185
## RunTime     -3.38479564  -1.98025030
## RunPulse    -0.61465745  -0.13214425
## MaxPulse     0.02906062   0.58075505
\end{verbatim}

\pagebreak

\subsubsection{\texorpdfstring{Appendix 2 \emph{R code and output for
bootstrapping code to provide confidence
intervals}}{Appendix 2 R code and output for bootstrapping code to provide confidence intervals}}\label{appendix-2-r-code-and-output-for-bootstrapping-code-to-provide-confidence-intervals}

\begin{Shaded}
\begin{Highlighting}[]
\CommentTok{#the bootstrapping function from groupwork}
\CommentTok{#install.packages("boot")}
\KeywordTok{library}\NormalTok{(boot)}

\NormalTok{lmBoot_par <-}\StringTok{ }\ControlFlowTok{function}\NormalTok{(inputData, nBoot)\{}
  \CommentTok{#Purpose: Generate a large number of linear regression beta coefficients using}
  \CommentTok{#         bootstrap methods.}
  \CommentTok{#Inputs: inputData: a dataframe containing the response variable, which must be }
  \CommentTok{#        in the first column of the dataframe, and the covariates of interest}
  \CommentTok{#        nBoot: the number of bootstrap samples to generate.}
  \CommentTok{#Outputs: BootResults: An arraycontaing the parameter estimates of each }
  \CommentTok{#         each bootstrap sample.}
  \CommentTok{#         ConfidenceIntervals: A matrix containing 95% confidence intervals }
  \CommentTok{#         for each parameter.}
  
  \CommentTok{#Calculate the number of observations in the dataset }
\NormalTok{  nObs <-}\StringTok{ }\KeywordTok{nrow}\NormalTok{(inputData)}
  
  \CommentTok{#Create the sample data with 1s for the intercept}
\NormalTok{  sampleData <-}\StringTok{ }\KeywordTok{as.matrix}\NormalTok{(}\KeywordTok{cbind}\NormalTok{(inputData[, }\DecValTok{1}\NormalTok{], }\DecValTok{1}\NormalTok{, inputData[, }\OperatorTok{-}\DecValTok{1}\NormalTok{]))}
  
  \CommentTok{#Set up parallisation}
\NormalTok{  nCores <-}\StringTok{ }\KeywordTok{detectCores}\NormalTok{()}
\NormalTok{  myClust <-}\StringTok{ }\KeywordTok{makeCluster}\NormalTok{(nCores }\OperatorTok{-}\StringTok{ }\DecValTok{1}\NormalTok{, }\DataTypeTok{type =} \StringTok{"PSOCK"}\NormalTok{)}
  \KeywordTok{registerDoParallel}\NormalTok{(myClust)}
  
  \CommentTok{# Create the samples}
\NormalTok{  bootSamples <-}\StringTok{ }\KeywordTok{matrix}\NormalTok{(}\KeywordTok{sample}\NormalTok{(}\DecValTok{1}\OperatorTok{:}\KeywordTok{nrow}\NormalTok{(inputData), nObs }\OperatorTok{*}\StringTok{ }\NormalTok{nBoot, }\DataTypeTok{replace =}\NormalTok{ T), }
                        \DataTypeTok{nrow =}\NormalTok{ nObs, }\DataTypeTok{ncol =}\NormalTok{ nBoot)}
  
  \CommentTok{#Use parallised sapply to apply bootLM to bootResults matrix}
\NormalTok{  bootResults <-}\StringTok{ }\KeywordTok{matrix}\NormalTok{(}\OtherTok{NA}\NormalTok{, nBoot, }\KeywordTok{ncol}\NormalTok{(sampleData[, }\OperatorTok{-}\DecValTok{1}\NormalTok{]))}
\NormalTok{  bootResults <-}\StringTok{ }\KeywordTok{parSapply}\NormalTok{(myClust, }\DecValTok{1}\OperatorTok{:}\NormalTok{nBoot, bootLM, }\DataTypeTok{inputData =}\NormalTok{ sampleData, }
                           \DataTypeTok{samples =}\NormalTok{ bootSamples)}
  
  \CommentTok{#Close parallisation}
  \KeywordTok{stopCluster}\NormalTok{(myClust)}
  
  \KeywordTok{return}\NormalTok{(}\KeywordTok{t}\NormalTok{(bootResults))}
\NormalTok{\}}

\CommentTok{#results of the bootstrapping results}
\CommentTok{#install.packages("parallel")}
\CommentTok{#install.packages("doParallel")}
\CommentTok{#install.packages("IPSUR")}
\KeywordTok{library}\NormalTok{(parallel)}
\KeywordTok{library}\NormalTok{(doParallel)}
\KeywordTok{library}\NormalTok{(IPSUR)}
\NormalTok{results <-}\StringTok{ }\KeywordTok{boot}\NormalTok{(}\DataTypeTok{data =}\NormalTok{ fitness, }\DataTypeTok{statistic =}\NormalTok{ lmBoot_par, }\DataTypeTok{R =} \DecValTok{10}\NormalTok{, }\DataTypeTok{formula =}\NormalTok{ alteredModel)}

\CommentTok{#provide confidence intervals}
\KeywordTok{boot.ci}\NormalTok{(results, }\DataTypeTok{type =} \StringTok{"basic"}\NormalTok{, }\DataTypeTok{index=}\DecValTok{1}\NormalTok{) }\CommentTok{#intercept}
\KeywordTok{boot.ci}\NormalTok{(results, }\DataTypeTok{type =} \StringTok{"basic"}\NormalTok{, }\DataTypeTok{index=}\DecValTok{2}\NormalTok{) }\CommentTok{#Age}
\KeywordTok{boot.ci}\NormalTok{(results, }\DataTypeTok{type =} \StringTok{"basic"}\NormalTok{, }\DataTypeTok{index=}\DecValTok{3}\NormalTok{) }\CommentTok{#Weight}
\KeywordTok{boot.ci}\NormalTok{(results, }\DataTypeTok{type =} \StringTok{"basic"}\NormalTok{, }\DataTypeTok{index=}\DecValTok{4}\NormalTok{) }\CommentTok{#RunTime}
\KeywordTok{boot.ci}\NormalTok{(results, }\DataTypeTok{type =} \StringTok{"basic"}\NormalTok{, }\DataTypeTok{index=}\DecValTok{5}\NormalTok{) }\CommentTok{#RunPulse}
\KeywordTok{boot.ci}\NormalTok{(results, }\DataTypeTok{type =} \StringTok{"basic"}\NormalTok{, }\DataTypeTok{index=}\DecValTok{6}\NormalTok{) }\CommentTok{#MaxPulse}
\end{Highlighting}
\end{Shaded}

\pagebreak

\subsubsection{\texorpdfstring{Appendix 3 \emph{R code and output for
randomisation
tests}}{Appendix 3 R code and output for randomisation tests}}\label{appendix-3-r-code-and-output-for-randomisation-tests}

\begin{Shaded}
\begin{Highlighting}[]
\KeywordTok{set.seed}\NormalTok{(}\DecValTok{180025784}\NormalTok{)}

\CommentTok{#remove the parameter "RestPulse" that is useless for the model}
\NormalTok{alteredFitness <-}\StringTok{ }\NormalTok{fitness }\OperatorTok\StringTok{ }\KeywordTok{select}\NormalTok{ (}\OperatorTok{-}\NormalTok{RestPulse)}

\CommentTok{#compute the slope of alteredModel}
\NormalTok{estimatedSlope <-}\StringTok{ }\KeywordTok{coef}\NormalTok{(alteredModel)[}\DecValTok{2}\NormalTok{]}

\CommentTok{#to store simulations}
\NormalTok{simResults <-}\StringTok{ }\KeywordTok{numeric}\NormalTok{(}\DecValTok{999}\NormalTok{)}

\NormalTok{simData <-}\StringTok{ }\NormalTok{alteredFitness}

\ControlFlowTok{for}\NormalTok{ (i }\ControlFlowTok{in} \DecValTok{1}\OperatorTok{:}\StringTok{ }\DecValTok{999}\NormalTok{)\{}
  \CommentTok{#shuffle the variables WRT Oxygen}
\NormalTok{  simData}\OperatorTok{$}\NormalTok{Oxygen <-}\StringTok{ }\KeywordTok{sample}\NormalTok{(alteredFitness}\OperatorTok{$}\NormalTok{Oxygen, }\DecValTok{31}\NormalTok{, }\DataTypeTok{replace =}\NormalTok{ T)}
  \CommentTok{#fit a model under H0}
\NormalTok{  simLM <-}\StringTok{ }\KeywordTok{lm}\NormalTok{ (Oxygen}\OperatorTok{~}\NormalTok{., }\DataTypeTok{data =}\NormalTok{ simData)}
  \CommentTok{#store the slope}
\NormalTok{  simResults[i] <-}\StringTok{ }\KeywordTok{coef}\NormalTok{(simLM)[}\DecValTok{2}\NormalTok{]}
\NormalTok{\}}

\CommentTok{#draw the histogram and line of estimatedSlope to compare}
\KeywordTok{hist}\NormalTok{(simResults, }\DataTypeTok{col =} \StringTok{"slateblue4"}\NormalTok{)}
\KeywordTok{abline}\NormalTok{(}\DataTypeTok{v =}\NormalTok{ estimatedSlope, }\DataTypeTok{lwd =} \DecValTok{3}\NormalTok{)}
\end{Highlighting}
\end{Shaded}

\includegraphics{180025784-MT5763-Project2_files/figure-latex/randomisation tests-1.pdf}

\begin{Shaded}
\begin{Highlighting}[]
\CommentTok{#reject H0 maybe}

\NormalTok{addEst <-}\StringTok{ }\KeywordTok{c}\NormalTok{(estimatedSlope, simResults)}

\NormalTok{locEst <-}\StringTok{ }\KeywordTok{c}\NormalTok{(}\DecValTok{1}\NormalTok{, }\KeywordTok{rep}\NormalTok{(}\DecValTok{0}\NormalTok{,}\DecValTok{999}\NormalTok{))}

\NormalTok{locEst <-}\StringTok{ }\NormalTok{locEst[}\KeywordTok{order}\NormalTok{(addEst)]}

\NormalTok{k <-}\StringTok{ }\KeywordTok{which}\NormalTok{(locEst }\OperatorTok{==}\StringTok{ }\DecValTok{1}\NormalTok{)}

\KeywordTok{min}\NormalTok{(k}\OperatorTok{/}\DecValTok{1000}\NormalTok{, }\DecValTok{1}\OperatorTok{-}\NormalTok{k}\OperatorTok{/}\DecValTok{1000}\NormalTok{)}\OperatorTok{*}\DecValTok{2}
\end{Highlighting}
\end{Shaded}

\begin{verbatim}
## [1] 0.31
\end{verbatim}

\begin{Shaded}
\begin{Highlighting}[]
\NormalTok{###partial relationships}
\KeywordTok{termplot}\NormalTok{(alteredModel, }\DataTypeTok{data =}\NormalTok{ alteredFitness,}\DataTypeTok{partial.resid =} \OtherTok{TRUE}\NormalTok{)}
\end{Highlighting}
\end{Shaded}

\includegraphics{180025784-MT5763-Project2_files/figure-latex/randomisation tests-2.pdf}
\includegraphics{180025784-MT5763-Project2_files/figure-latex/randomisation tests-3.pdf}
\includegraphics{180025784-MT5763-Project2_files/figure-latex/randomisation tests-4.pdf}
\includegraphics{180025784-MT5763-Project2_files/figure-latex/randomisation tests-5.pdf}
\includegraphics{180025784-MT5763-Project2_files/figure-latex/randomisation tests-6.pdf}


\end{document}
